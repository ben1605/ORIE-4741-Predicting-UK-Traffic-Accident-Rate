\documentclass{article}
\usepackage[utf8]{inputenc}

\title {\textbf{Proposal}}


\begin{document}
\date{}\maketitle

{\textbf{\fontsize{15}{30}\selectfont How to predict traffic accident rate in UK?}

}

\n
{\fontsize{14}{16}\selectfont 
{We would like to estimate UK?s traffic accident rate, and found the datasets from UK?s Department for Transport. There are two files. One records each of the traffic accidents happened in UK with details from 2005 to 2014,with the exception that there is a loss of data happened in 2008. On the other hand, the other file contains the traffic flow in over ten regions in UK from 2000 to 2015. 


Using the predicted accident rate, the government could decide how much police force to be deployed in each area. This allows public resources to be used more efficiently. Furthermore, the government could take a deeper look at the areas with especially high accident rate and make improvements to reduce the accident rate. }

{\fontsize{14}{16}\selectfont 
\\
We believe that the data set we used can effectively answer the question. First of all the data set includes almost all the traffic accidents happened in UK from 2000 to 2015, so it is convincing to use these data to predict the future traffic accidents in UK. Secondly we use the data set to separate UK in to around 18,000 areas, so we predict the traffic accidents in each area using the features of these areas from the data set. Our main idea is to calculate how many traffic accidents happened every year in every area, then try to find the relationship between the number of traffic accidents in every area and the features of every area so that we can establish our hypothesis.
}}
\end{document}
